%%% S E C T I O N - - - - - - - - - - - - - - - - - - - - - - -
\section{Motivations}\label{S_motivations}
%{{{

The Drake equation \citep{drake_intelligent_1962} offers a very
helpful educated guess, a rational set of lenses --the factors in the
equation-- through which to look at future contacts with a
technologically advanced civilization in the Milky Way.
%
The equation quantifies the number of civilizations from whom we might
receive an electromagnetic signal, using a number of factors that have
been widely discussed in the literature and whose estimates are
updated constantly.
%
A comprehensive review and analysis of each term of the equation is
presented in \citet{vakoch_drake_2015}.
%
Some optimistic estimates from the Drake equation contrast with the so
called Fermi paradox, a statement about the contradiction between the
apparent abundance of life in the Galaxy and its lack of evidence
\citep[e.g. ][]{hart_explanation_1975, brin_great_1983,
barlow_galactic_2012, forgan_galactic_2016, vanhouten_isthere_2017,
Sotos_biotechnology_2019, carroll_nellemback_fermi_2019}.
%
There are many proposals aimed at solving this paradox, which make use
of statistical \citep{solomonides_probabilistic_2016,
vanhouten_isthere_2017, horvat_calculating_2007,
maccone_statistical_2015} or stochastic approaches
\citep{forgan_numerical_2009, bloetscher_using_2019,
glade_stochastic_2011, forgan_numerical_2010}.
%
Regarding the Drake equation, analytical interpretations
\citep{prantzos_joint_2013,
smith_broadcasting_2009} or reformulations
\citep[][and references therein]{burchell_whither_2006} have also been
proposed.
%
The lack of contacts could be explained by the difficulty of life to
appear due to astrophysical explanations
\citep{annis_astrophysical_1999}.
%
In general, also more speculative proposals
\citep{barlow_galactic_2013, lampton_information_2013,
conway_three_2018, forgan_galactic_2017} have been discussed.
% 
\Fpagebreak
%
We stress the fact that the only observation that can be stated with
certainty is that for the number of years projects for the Search of
Extraterrestrial Intelligences (SETI) have been working, we have not
received any signal using the technical possibilities and the
conditions established by SETI projects \citep{tarter_search_2001}.
%
The lack of detection has also inspired to a number of alternative
proposals for new SETI strategies \citep{forgan_exoplanet_2017,
balbi_impact_2018, loeb_eavesdropping_2006, maccone_KLT_2010,
tarter_advancing_2009, enriquez_breakthrough_2017, loeb_relative_2016,
maccone_SETI_2011,  lingam_relative_2019, wright_theGsearch_2015,
maccone_SETI_2013, maccone_lognormals_2014, harp_application_2018,
forgan_possibility_2013, forgan_galactic_2017, funes_searching_2019}.
%
The Drake equation is a key tool to organize the discussion around the
problem of the abundance of civilizations in the Galaxy
\citep{hinkel_interdisciplinary_2019}.
%
However, the uncertainties in the factors of this equation make it
less prone to a formal application in order to define searching
strategies or to compute the actual number of extraterrestrial
intelligences.
%
The optimistic estimations from the Drake equation also contrast with
the SETI initiatives, noticeable overshadowed by the fact that despite
decades of effort, not a single extraterrestrial intelligence, if any,
has been detected.
%
\citet{prantzos_joint_2013} proposes a unified framework for a joint
analysis of the Drake equation and the Fermi paradox, concluding that
for sufficiently long-lived civilizations, colonization of the Galaxy
is the only reasonable option to gain knowledge about other life
forms.
%
\citet{haqq-misra_drake_2017} discuss the dependence of the Drake
equation parameters on the spectral type of the host star and the time
since the Galaxy formed and examine trajectories for the emergence of
communicative civilizations over the history of the Galaxy.
%
Some modifications to the original idea of the equation have tried to
imprint a stochastic nature, to propose a probabilistic approach, or
to consider the temporal structure which is missing in the equation.
%
Temporal aspects of the distribution of communicating civilizations
and their contacts have also been explored by several authors
\citep{fogg_temporal_1987, forgan_spatiotemporal_2011,
balbi_impact_2018, balb_spatiotemporal_2018, horvat_impact_2011}, as
well as efforts on considering the stochastic nature of the Drake
equation \citep{glade_stochastic_2011}.
%
The simulation approach has also been considered
\citep{forgan_evaluating_2015, vukotic_grandeur_2016,
murante_simulating_2015, forgan_numerical_2009, forgan_galactic_2017,
ramirez_new_2017}, although with a similar problem that consists on
the large number of parameters which are either unknown or largely
uncertain.
%
\citet{grimaldi_signal_2017} consider a statistical model for the
probability of the Earth making contact with other intelligent
civilizations, taking into account the finite lifetime of signal
emitters, and based on the fractional volume occupied by all signals
reaching our planet.



In a recent work, \citep{balbi_impact_2018} use a statistical model to
analyze the occurrence of causal contacts between civilizations in the
Galaxy.
%
The author highlights the effect of evolutionary processes when
attempting to estimate the number of communicating civilizations that
might be in causal contact with an observer located on the Earth.
%
\citet{cirkovic_temporal_2004} also emphasize the lack of temporal
structure in the Drake equation.
%
\citet{bloetscher_using_2019} uses a probabilistic approach, though
still strongly motivated by the Drake equation, to obtain a
probabilistic measure for the number of civilizations in the Galaxy.
%
To that end, perform MonteCarlo Markov Chains over each factor of the
Drake equation, and combine the means to obtain a probabilistic
result.
%
It is worth mentioning that the author proposes a log-normal target
distribution to compute the posterior probabilities.
%
%This approach, which assigns probability distributions to the Drake
%factors.
%
This study concludes that there is a very low probability for Earth to
make contact with another extraterrestrial civilization.    
%
\citet{smith_broadcasting_2009} uses an analytical model to compute
the probabilities of contact between two randomly located
civilizations, and the waiting time for the first contact, assuming a
fixed maximum broadcasting distance.
% 
Using a different approach, \citet{balbi_impact_2018} investigate the
chance of communicating civilizations making causal contact within a
volume surrounding the location of the Earth, using a statistical
model to compute the occurrence of causal contacts.
%
The author stresses the fact that the causal contact requirement
involves the distance between civilizations, their lifespan and their
times of appearance.  This is important since the time it takes the
light to travel across the Galaxy is much lesser than its lifetime.
%
\citet{balbi_impact_2018} fix the total number of civilizations and
explore the parameter space that comprise three parameters, namely,
the distance to the Earth, the time of appearance and the lifespan of
the communicating civilization.
%
Each of these three variables are drawn from a random distribution.
The distribution of the distances results from a uniform distribution
of civilization within the plane of the Galaxy.
%
For the distribution of the characteristic time of appearance, the
author explores and exponential and a truncated Gaussian
distributions. For the distribution of the lifespans, the author
chooses an exponential distribution.
%
%The radius of influence is set to 1000 lyr.
%
\citet{balbi_impact_2018} finds that the fraction of communicating
civilizations vary with the choice of the statistical distribution for
the time of appearance. For all analyzed distributions, the fraction
corresponding to a mean lifetime of 10$^9$ years and for a maximum
radius for the detection of the signal of 1000 yr, is roughly 0.1.
%
In another probabilistic approach,
\citet{vukotic_astrobiological_2012} propose a probabilistic cellular
automata (PCA) modeling. In this framework, a complex system is
modeled by a lattice of cells which evolve at discrete time steps,
according to transition rules that take into account for each cell the
states of its neighbor cells.
%
The authors implement a PCA model to a network of cells which
represent life complexity on a 2-dimensional annular ring resembling
the Galactic Habitable Zone (GHZ). 
%
The authors set the GHZ between a minimum radius of 6 kpc and a
maximum radius of 10 kpc. Within this framework, the authors also make
several MonteCarlo simulations and analyze ensemble-averaged results.
%
This work aims at analyzing the evolution of life, although it does
not account for the network of causal contacts among technological
civilizations.




In this work we address the problem of the temporal and spatial
structure of the distribution of communicating civilizations, by
exploring the hypothesis space over a minimal set of parameters.
%
We propose to avoid the frequentist approach of the
Drake equation and to explore a parameter space, where instead of
computing a final number, we provide a statistical distribution that
gives conditional probabilities.
%
This is an exploratory analysis that aims at providing a numerical
tool to discuss not the several theoretical problems summarized by the
Drake equation factors, but the different scenarios on the basis of
statistical heuristics.
%
The approach proposed here should be considered as a compromise
between the uncertainties of the frequentist approach and the detailed
recipes required on the simulation approach.
%
We then provide a numerical framework to explore via simulations a
parameter space of unknown observables in order to discuss different
scenarios and their consequences in terms of the probability of
making contacts.
%
The method we use for simulating a stochastic process is an
approximation that allows to study the behavior of complex systems,
by considering a sequence of well defined discrete events.
%
The simulation is carried out by following all the variables that
describe and constitute the state of the system, and the evolution of
the process is described as a set of changes in that state.
%
In this context, an event produces a specific change in the state,
that can be triggered by random variables that encode the stochastic
nature of the physical phenomenon.
%
For example, when a new contact is produced between two entities in
the simulated galaxy, the numbers of active communication lines and of
active communicated nodes change.
%
The process involves following the changes on the state of the
system, defining the initial and final states, defining a method that
allows to keep track of the time progress in steps, and maintaining a
list of relevant events.
%
In Sec.~\ref{S_methods} we introduce the methods and discuss the
candidate distributions for the statistical aspects of the times
involved in the communication process.
%
Then, we present our results in Sec~\ref{S_results}, with special
emphasis on the statistical distributions of the duration of causal
contacts in one or both directions, the possible differences on the
position of a civilization on the Galaxy and the distribution of time
intervals for the waiting of the first contact, always as a function
of the three simulation parameters.
%
Finally, in Sec.~\ref{S_discussion} we discuss our results and future
research directions.

%}}}


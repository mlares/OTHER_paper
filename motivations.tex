%%% S E C T I O N - - - - - - - - - - - - - - - - - - - - - - -
\section{Motivations}\label{S_motivations}
%{{{

The Drake equation \citep{drake_intelligent_1962}
offers a very helpful educated guess,
a rational set of lenses --the factors in the equation-- through which
to look at future contacts with a technologically advanced
civilization in the Milky Way.
%
The equation quantifies the number of
civilizations from whom we might receive an electromagnetic signal.
%
For a comprehensive review and analysis of each term of the equation
see \citet{vakoch_drake_2015}.
%
\citet{prantzos_joint_2013}
proposes a unified framework
for a joint analysis of the Drake equation and the Fermi paradox
concluding that for sufficiently long-lived civilizations,
colonization of the Galaxy is the only reasonable option to gain
knowledge about other life forms.
%
\citet{haqq-misra_drake_2017}
discuss the dependence of the Drake equation parameters on the
spectral type of the host star and the time since the galaxy formed
and examine trajectories for the emergence of communicative
civilizations over the history of the galaxy.
%
The abundance of
intelligent civilizations in the galaxy is a longstanding question,
which is often conceptualized as the problem of the lack of received
communication or the Fermi paradox 
\citep{barlow_galactic_2012, Sotos_biotechnology_2019,
forgan_galactic_2016}.



There are many proposals aimed at solving this paradox, which make use
of statistical \citep{solomonides_probabilistic_2016,
vanhouten_isthere_2017, horvat_calculating_2007,
maccone_statistical_2015} or stochastic approaches
\citep{forgan_numerical_2009, bloetscher_using_2019,
glade_stochastic_2011, forgan_numerical_2010}, analytical
interpretations of the Drake equation \citep{prantzos_joint_2013,
smith_broadcasting_2009} or more speculative proposals
\citep{barlow_galactic_2013, lampton_information_2013,
conway_three_2018, forgan_galactic_2017}.
%
%%%%% agregar Carroll-Nellenback, 2019).
%
However, the uncertainties in the factors of the Drake equation make
it less prone to a formal application in order to define searching
strategies or to compute the actual number of ETIs, but is a key tool
to organize the discussion around the problem of the population of
ETIs in the Galaxy \citep{hinkel_interdisciplinary_2019}, which is
noticeable dominated by the fact that despite decades of effort not a
single ETI, if any, has been detected so far.
%
Some modifications to the original idea of the equation have tried to
imprint a stochastic nature, or to propose a probabilistic approach,
or to consider the temporal structure which is missing in the
equation.


\Fpagebreak


Temporal aspects of the distribution of communicating civilizations
and their contacts have also been explored by several authors
\citep{fogg_temporal_1987, forgan_spatiotemporal_2011,
balbi_impact_2018, balb_spatiotemporal_2018, horvat_impact_2011},
%The Impact of the Temporal Distribution of Communicating
%Civilizations on Their Detectability Temporal dispersion of the
%emergence of intelligence: an inter-arrival time analysis The
%spatiotemporal aspects of SETI
%
as well as efforts on considering the stochastic nature of the Drake
equation \citep{glade_stochastic_2011}.
%
The simulation approach has also been considered
\citep{forgan_evaluating_2015, vukotic_grandeur_2016,
murante_simulating_2015, forgan_numerical_2009, forgan_galactic_2017},
although with a similar problem that consists on the large number of
parameters which are either unknown or largely uncertain.
%
%New numerical determination of habitability in the Galaxy: the SETI
%connection
%
%Since the factors in the Drake equation are uncertain,
%
In this work, we propose to avoid the frequentist approach of the
Drake equation and to explore a parameter space, where instead of
computing a final number, we provide a statistical distribution that
gives conditional probabilities.
%
This is an exploratory analysis that aims at providing a numerical
tool to discuss not the several theoretical problems summarized by the
Drake equation factors, but the different scenarios on the basis of
statistical heuristics.
%
The approach proposed here shold be considered as a compromise between
the uncertainties of the frequentist approach and the detailed recipes
required on the simulation approach.
%
We then provide a numerical framework to explore via simulations a
parameter space of unknown observables in order to discuss different
scenarios and their consecuences in terms of the probability con
making contacts.
%
The only fact that can be stated with certainty is that for the number
of years SETI projects have been working we have not received any
signal, at least within the technical posibilities and the conditions
stablished by SETI \citep{tarter_search_2001}.
%
This arguably astonishing result has lead to the Fermi paradox, which
states an apparent contradiction between the high estimation of the
abundance of life in the Galaxy and the lack of their detection
\citep{vanhouten_isthere_2017}.
%
The lack of detection has also ispired to a number of alternative
proposals for new SETI strategies \citep{forgan_exoplanet_2017,
balbi_impact_2018, loeb_eavesdropping_2006, maccone_KLT_2010,
tarter_advancing_2009, enriquez_breakthrough_2017, loeb_relative_2016,
maccone_SETI_2011,  lingam_relative_2019, wright_theGsearch_2015,
maccone_SETI_2013, maccone_lognormals_2014, harp_application_2018,
forgan_possibility_2013, forgan_galactic_2017, funes_searching_2019}.
 


The discrete events method for simulating a stochastic process is an
approximation that allows to study the behaviour of complex systems,
by considering a sequence of well defined discrete events.
%
The simulation is carried out by following all the variables that
describe the system, that constitute the state of the system.
%
The evolution of the process, then, is described as a set of changes
in the state of the system.
%
In this context, an event produces a specific change in the state,
that can be triggered by random variables that encode the stochastic
nature of the physical phenomenon.
%
The process involves following the changes on the state of the system,
definig the initial and final states, defining a method that allows to
keep track of time progress, and maintaining a list of relevant
events. (REESCRIBIR!!)


In a recent work, \citep{balbi_impact_2018} use a statistical model to
analyze the occurrence of causal contacts between civilizations in the
Galaxy.
%
The author highlights the effect of evolutionary processes when
attempting to estimate the number of communicating civilizations that
might be in causal contact with an observer located on the Earth.
%
\citet{cirkovic_temporal_2004} also emphasize the lack of temporal
structure in the Drake equation.



%While the Drake equation remains a very useful
%guide in identifying and analyzing the various factors in-
%volved in the problem and their relative probabilities, our
%approach could provide a better framework with which to
%perform a statistical analysis when taking into account evo-
%lutionary processes. In particular, we have shown that the
%Drake equation can misestimate the number of detectable
%communicating species if spatial and temporal dependencies
%are important. This should suggest some caution in the un-
%critical application of such an equation. In this paper, we offer
%an exemplificative illustration of the effect of different
%probability distributions for the typical timescales involved
%in the problem. Ideally, it would be desirable to arrive at a
%mathematical description of such distributions based on a
%modeling of the underlying astrophysical and biological
%processes. This points out a possible future direction of in-
%vestigation in which an integrated evolutionary (and coevo-
%lutionary) approach to the problem would be a central focus
%of interest (for more on this perspective, see, e.g., Chaisson


\citet{bloetscher_using_2019} uses a probabilistic approach, though
still strongly motivated by the Drake equation, to obtain a
probabilistic measure for the number of civilizations in the Galaxy.
%
To that end, perform MonteCarlo Markov Chains over each factor of the
Drake equation, and combine the means to obtain a probabilistic
result.
%
It is worth mentioning that the author proposes a log-normal target
distribution to compute the posterior probabilities.
%
This approach, which assigns probability distributions to the Drake
factors.
%
This study concludes that there is a very low probability for Earth to
make contact with another extraterrestrail civilization.    
%
\citet{smith_broadcasting_2009} uses a analytical model to compute the
probabilities of contact between two randomly located civilizations,
and the waiting time for the first contact, assuming a fixed maximum
broadcasting distance.
% 
\citet{balbi_impact_2018} investigate the chance of communicating
civilizations making causal contact within a volume surrounding the
location of the Earth, using a statistical model to compute the
occurrence of causal contacts. The author stresses the fact that the
causal contact requirement involves the distance between
civilizations, their lifespan and their times of appeareance.  This is
important since the time it takes the light to travel across the
Galaxy is much lesser than the lifetime of the Galaxy.
%
\citet{balbi_impact_2018} fix the total number of civilizations and
explore the parameter space that comprise three parameters, namely,
the distance to the Earth, the time of appeareance and the lifespan of
the communicating civilization. Each of these three variables are
drawn from a random distribution. The distribution of the distances
results from a uniform distribution of civilization within the plane
of the Galaxy. For the distribution of the characteristic time of
appeareance, the author explores and exponential and a trucated
Gaussian distributions. For the distribution of the lifespans, the
author choses an exponential distribution. The radius of influence is
set to 1000 ly.
%
\citet{balbi_impact_2018} finds that the fraction of communicating
civilizations vary with the choice of the statistical distribution for
the time of appeareance. For all analyzed distributions, the fraction
corresponding to a mean lifetime of 10$^9$ years and for a maximu
radius for the detection of the signal of 1000 ys, is roughly 0.1.
%
\citet{vukotic_astrobiological_2012} propose a simulation approach
based on probabilistic cellular automata (PCA) modeling. In this
framework, a complex system is modeled by a lattice of cells which
evolve at discrete time steps, acording to transition rules that take
into account for each cell the states of its neighbour cells.
%
The authors implement a PCA model to a network of cells which
represent life complexity on a 2-dimensional annular ring resembling
the GHZ. The authors set the GHZ between a minimum radius of 6 kpc and
a maximum radius of 10 kpc. Within this framework, the authors also
make several MonteCarlo simulations and analyze ensemble-averaged
results.
%
This work aims at analyzing the evolution of life, although it does
not account for the network of causal contacts among technological
civilizations.

%- - - -

In this work we address the problem of the temporal and spatial
structure of the distribution of communicating civilizations, by
exploring the hypothesis space over a minimal set of parameters.
%
In Sec.~\ref{S_methods} we introduce the methods and discuss the
candidate distributions for the statistical aspects of the times
involved in the communication process.
%
Then, we present our results in Sec~\ref{S_results}, with special
emphasis on the statistical distributions of the duration of causal
contacts in one or both directions, the possible differences on the
position of a civilization on the Galaxy and the distribution of time
intervals for the waiting of the first contact, always as a function
of the three simulation parameters.
%
Finally, in Sec.~\ref{S_discussion} we discuss our results and future
research directions.


% HACER DISCUSION FUERTE DE ESTOS PAPERS: \cite{balbi_impact_2018,cirkovic_temporal_2004,grimaldi_signal_2017,smith_broadcasting_2009}

%}}}

